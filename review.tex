\documentclass[11pt]{article}
\usepackage{standalone}
\usepackage[margin=0.75in, headheight=20pt]{geometry}

\usepackage{amsmath}
\usepackage{amsfonts}
\usepackage{mathtools}

\usepackage[utf8]{inputenc}
\usepackage[english]{babel}
\setlength{\parindent}{2em}
\setlength{\parskip}{.25em}
\renewcommand{\baselinestretch}{1.0}

\usepackage{fancyhdr}
\pagestyle{fancy}
\rhead{ Clarke | Blostein | Queen's University}
\renewcommand{\headrulewidth}{0.4pt}
\renewcommand{\footrulewidth}{0.4pt}

\usepackage{courier}

\usepackage[]{algorithm2e}
\usepackage{mathrsfs}
\usepackage{eufrak}

\usepackage{etoolbox}
\patchcmd{\thebibliography}{\chapter*}{\section*}{}{}




\title{Semi-orthogonal User Selection Schemes for MU-MIMO Systems\\ Literature Review and Algorithm Development}
\author{J.E. Clarke, Dr. S.D. Blostein | Queen's University}
\date{Summer, 2018}

\begin{document}
	\maketitle
	\newpage
    \section{Literature Review}
    	%\subsection{Jeong et al.\cite{7354607}: Similar to Oni's work, no orthogonality consideration in user selection}
        %    Work relating MU-MIMO group selection to 802.11ac association has been done by Jeong et al \cite{7354607}. This work assumes multi-antenna APs, and single-antenna STAs. A MU-MIMO scenario similar to 802.11ac downlink is considered. The work does not consider semi-orthogonal group selection when it does STA group selection. Rather, grouping is constrained such that group size is less than or equal to the number of AP antennas, and that groups associated with separate APs are mutually exclusive (ie. an STA cannot belong to two groups that are associated with different APs).	
        %\subsection{Xie and Lok \cite{7510766}: Distributed method for solving the association problem based on auctioning}
        %    This work sets out to solve the association problem in a distributed fashion based on auctioning. This paper is based on the uplink rather than the downlink. The auction is based on resources determined from channel sounding. The argument in the paper is that the channel sounding is easy to do in the uplink in the AP. Although this paper is likely a good resource to consider when developing a distributed algorithm for solving the association problem, there is likely going to need to be some thought put into how the channel sounding is going to be done effectively in the downlink.
    
        \subsection{Swannack \cite{1549555}: Semi-orthogonal user selection based on $\epsilon$-orthogonality condition}
            

            
        %\subsection{Yoo and Goldstein \cite{1603708}:Semi-orthogonal user selection and ZFBF }
        %    The focus of this study is motivated by developing an alternative to dirty paper coding (DPC) that is less computationally expensive. It is postulated that DPC is an optimal scheme in terms of throughput maximization in so far as its reduction of interference between users in a MU-MIMO network. The drawback of this coding scheme is it's high computational complexity. Authors present zero-forcing beamforming (ZFBF) scheme in coordination with semi-orthogonal user selection scheme as a computationally simplistic alternative to DPC.

Authors argue that as the number of users grows, the performance of SUS-ZFBF approaches DPC. The reason for this asymptotic relationship are two-fold. Firstly, assuming a Rayleigh-fading channel, the transmitter has more choices to avoid poor channel conditions. The user with the best SNR exhibits an SNR a factor of $\text{log}(K)$ higher than the average user SNR (where $K$ is the number of users present in the environment). Secondly, the diversity provided by user separation offers increased diversity. Separation of users provides more isolation between users, resulting in lower interference, and more spatially independent choices for the user to transmit to, thus enhancing overall capacity.

Paper develops good channel model. Rates for TDMA, DPC, and ZFBF schemes are well-developed.

It is identified that optimal SUS requires an exhaustive search on the set of STAs present. This search becomes in feasible as this set gets large. Therefore, the authors offer an algorithm with reduced complexity to find a sub-optimal SUS scheme. The scheme presented depends on an orthogonality variable, $\alpha$, that permits a degree of flexibility when developing SUS groups (similar to the $\epsilon$ variable in Swannack \cite{1549555}). The algorithm presented only identifies one SUS group. Extension of this algorithm is required for our work. 

Note that \cite{1603708} goes on to develop a ZFBF scheme based on the SUS parameters. The ZFBF scheme presented decomposes the channel matrex of the SUS group, $\textbf{H}(\mathcal{S})$, into diagonal, lower triangle, and unitary matrices which depend on $\alpha$, $\textbf{g}_k$. Once the decomposition is performed, a realizable ZFBF scheme is presented. Performance is lower-bounded based on orthogonality arguments.

Complexity analysis is provided, which is likely useful in justifying a SUS scheme opposed to other schemes (such as DPC, TDMA etc.).

Results show that there is a convex relationship between $\alpha$ and sum rate. This makes sense since there is a sweet spot when trading off quantity vs. quality associated with $\alpha$.

The work goes on to consider fairness in its scheduling schemes. This will likely be of interest at some point.

    
    %\section{Proposed Semi-orthogonal User Selection Algorithms}
    %    Solutions to the semi-orthogonal user selection (SUS) problem can be roughly classified into three categories. The first category is the identification of a single SUS group. In this context, assume an AP is capable of providing service to a group of STAs represented by a set $\mathcal{C}$. A SUS group, $\mathcal{S} \subset \mathcal{C}$ may only contain part of $\mathcal{C}$. Therefore there are many such solutions that exist in $\mathcal{C}$. A second solution the SUS problem is by finding multiple SUS groups such that the collection of these groups partition the set $\mathcal{C}$. That is, the partition, $\mathscr{S}$ is composed of SUS groups $\mathcal{S}_l\ :\ l = 1,2,\ldots L$ such that it fully partitions $\mathcal{C}$. Similar to the first case, this solution is not unique. There may be many such partitions $\mathscr{S}$ of $\mathcal{C}$. Finally, an exhaustive solution to the problem may be conceived by describing all the potential partitions of $\mathcal{C}$. That is, the collection of all possible partitions on $\mathcal{C}$, $\mathfrak{S}$, is a set of partitions $\mathscr{S}_n\ : \ n=1,2,\ldots N$ on $\mathcal{C}$, which, in turn, is a set of SUS groups $\mathcal{S}_l \ :\ l = 1,2,\ldots L$. The size of the sets, $L,N$, depend on $\vert \mathcal{C}\vert$ and the strictness of the orthogonality requirement between SUS groups.


    %    \subsection{Algorithm based on Yoo and Goldstein \cite{1603708}}
    %        The algorithm developed in \cite{1603708} only identifies one group: it does not identify all the potential SUS groups. Therefore, an extension of this algorithm is required for our study. At minimum, we must have a set of viable SUS groups that belong to one AP. A more exhaustive search would return all the viable sets of SUS groups. 

The utility of such an exhaustive search is limited, especially when considering the computational cost of executing such a search.
Assume the following definitions:
\begin{itemize}
    \item Let $\mathcal{C}$ denote the set of STAs available to form SUS groups
    \item Let $K = \vert \mathcal{C} \vert $
    \item Let $k \in \mathcal{C}$ be an arbitrary STA
    \item Let $M = $ the number of antennas at an arbitrary AP, $a$
    \item Let $\textbf{h}_k$ be the $M \times 1$ channel vector for $k$
    \item Let $i$ be the variable of iteration
    \item Let $\mathcal{T}_i$ be the set of STAs who are semi-orthogonal candidates to be added to an SUS group at the $i^{th}$ iteration
    \item Let $\mathscr{S}$ be a partition on $\mathcal{C}$. It is a set containing the SUS groups
    \item Let $\mathcal{S}_l \in \mathscr{S}$ be the $l^{th}$ sub-optimal SUS group. The elements of $\mathcal{S}$ are STAs.
    \item Let $\alpha$ be the scalar representing the orthogonality requirement of the SUS group
\end{itemize}


\textbf{Algorithm from \cite{1603708}:}
\begin{enumerate}
    \item Initialization:
    \begin{subequations}
        \begin{align}
            \mathcal{T}_1 &= \lbrace 1 \ldots K \rbrace = \mathcal{C}\\
            i &= 1\\
            \mathcal{S}_1 &= \lbrace \emptyset \rbrace
        \end{align}
    \end{subequations}
    Start by populating the candidate set with the entire set of STAs available to the AP. Initialize the SUS group to be an empty set.
    \item Calculate orthogonal component of $\textbf{h}_k$:\\
    For $\forall \ k \in \mathcal{T}_i$:
    \begin{subequations}
        \begin{align}
            \textbf{g}_k &= \textbf{h}_k - \sum_{j=1}^{i-1}\frac{\textbf{h}_k\textbf{g}_j^*}{\Vert \textbf{g}_j \Vert^2}\textbf{g}_j \\
            &= \textbf{h}_k\bigg(\textbf{I}-\sum_{j=1}^{i-1}\frac{\textbf{g}_k^*\textbf{g}_j}{\Vert \textbf{g}_j \Vert^2}\bigg)
        \end{align}
    \end{subequations}
    This expression represents the projection of the channel vector $\textbf{h}_k$ onto the orthogonal complement of the span of space formed by $\text{span}(\textbf{g}_1\ldots \textbf{g}_{i-1})$. Initially, $\textbf{g}_k = \textbf{h}_k \ \forall \ k \in \mathcal{T}_1$. Then as the algorithm progresses, $\textbf{g}_k$ represents an inverse degree of orthogonality. This can be seen in Eq. (2a): as the second term $\longrightarrow 0$, representing orthogonality between $\textbf{h}_k$ and $\textbf{g}_j$, $\textbf{g}_k$ gets large.
    
    \item Select the best STA representing the highest degree of orthogonality:
    \begin{subequations}
        \begin{align}
            \pi(i) &= \arg\max_{k\in \mathcal{T}_i}\Vert \textbf{g}_k \Vert\\
            \mathcal{S}_1 &= \mathcal{S}_1 \cup \pi(i)\\
            \textbf{h}_i &= \textbf{h}_{\pi(i)}\\
            \textbf{g}_i &= \textbf{g}_{\pi(i)}
        \end{align}
    \end{subequations}
    In this step, the STA with the hightest degree of orthogonality to the remaining candidates is selected as the next STA to be added to the SUS group, $\mathcal{S}_o$. Channel and orthogonality vectors are updated to reflect this selection.
    
    \item Update the set of candidate STAs based on semi-orthogonality constraint:\\
    If $\vert \mathcal{S}_1 \vert < M$, then:
    \begin{subequations}
        \begin{align}
            \mathcal{T}_{i+1} &= \bigg\lbrace k\in \mathcal{T}_i,\ k\neq \pi(i)\ \big \vert \ \frac{\vert \textbf{h}_k\textbf{g}_i^*\vert}{\Vert \textbf{h}_k \Vert \Vert \textbf{g}_i\Vert} < \alpha  \bigg\rbrace \\
            i &= i+1
        \end{align}
    \end{subequations}
    Else: Terminate algorithm.
    The set of candidate STAs is updated for the next iteration of the algorithm by  removing any STAs that do not meet a minimum orthogonality requirement, determined by $\alpha$.
\end{enumerate}


\textbf{Extension of Algorithm}\\
The algorithm developed here extends the SUS algorithm developed in \cite{1603708}. The algorithm here takes the set of viable STAs, $\mathcal{C}$, the channel information for each STA $k \in \mathcal{C}$, $\textbf{h}_k$, and a orthogonality threshold, $\alpha$. The algorithm takes this information and partitions $\mathcal{C}$ into semi-orthogonal subsets, $\mathscr{S}$.

The solution provided by this algorithm is only one of many potential partitions on $\mathcal{C}$. That is, the solution provided here is not exhaustive, nor is it unique.

The algorithm essentially wraps the algorithm developed in \cite{1603708} in additional loop that checks to see that all the STS $\in \mathcal{C}$ have been assigned to a SUS group. The algorithm does not guarantee that all groups are the same size; however, it does guarantee that the SUS group size is less than the number of antennas at the AP.

\begin{algorithm}[H]
 \KwData{\\$\mathcal{C}$ (set of viable STAs);\\$\textbf{h}_k,\ \forall \ k\in\mathcal{C}$ (channel information for each STA);\\
 $\alpha$ (orthogonality requirement)}
 \KwResult{$\mathscr{S}$ (SUS partition on $\mathcal{C}$)}
 initialization;\\
        $\mathcal{L} = \lbrace 1 \ldots K \rbrace = \mathcal{C}$;\\
        $\mathcal{T}_1 = \lbrace 1 \ldots K \rbrace = \mathcal{C}$;\\
        $\mathcal{S}_1 = \lbrace \emptyset \rbrace$;\\
        $i = 1$;\\
        $l = 1$;\\

 \While{$\mathscr{L} \neq \lbrace \emptyset \rbrace$}{
     \While{$\vert \mathcal{S}_l \vert < M$ and $\mathcal{T}_i\neq\lbrace \emptyset \rbrace$}{
        \For{$\forall\ k \in \mathcal{T}_i$}{
            $\textbf{g}_k = \textbf{h}_k - \sum_{j=1}^{i-1}\frac{\textbf{h}_k\textbf{g}_j^*}{\Vert \textbf{g}_j \Vert^2}\textbf{g}_j$; \\
            %&= \textbf{h}_k\bigg(\textbf{I}-\sum_{j=1}^{i-1}\frac{\textbf{g}_k^*\textbf{g}_j}{\Vert \textbf{g}_j \Vert^2}\bigg)
        }
        $\pi(i) = \arg\max_{k\in \mathcal{T}_i}\Vert \textbf{g}_k \Vert$;\\
        $\mathcal{S}_1 = \mathcal{S}_l \cup \pi(i)$;\\
        $\textbf{h}_i = \textbf{h}_{\pi(i)}$;\\
        $\textbf{g}_i = \textbf{g}_{\pi(i)}$;\\
        //Add remaining STA to candidate set for SUS group if it meets SUS threshold\\
        //Otherwise add to the leftover set $\mathcal{L}$\\
        \For{$\forall \ k \in \mathcal{T}_i$,$\ k\neq \pi(i)$}{
            \eIf{$\frac{\vert \textbf{h}_k\textbf{g}_i^*\vert}{\Vert \textbf{h}_k \Vert \Vert \textbf{g}_i\Vert} < \alpha$}{
                $k \in \mathcal{T}_{i+1}$;\\
                $k \notin \mathcal{L}$;\\
            }{
                $k \in \mathcal{L}$;\\
                $k \notin \mathcal{T}_{i+1}$;\\
            }  
        }
        $i = i + 1$;\\
    }
    $l = l+1$;
 }
 \caption{Semi-orthogonal user selection algorithm}
\end{algorithm}
    %    \subsection{Exhaustive tree-based algorithm}
    %        The following algorithm finds all possible partitions $\mathfrak{S}$ on $\mathcal{C}$ based on forming STAs into trees on the basis of orthogonality.

It is assumed that any given STA cannot belong to more than one SUS group. A scheduling scheme that allocates temporal resources to the groups developed here is not discussed. The algorithm could be modified to allow an STA to belong to more than one SUS group. This would likely make scheduling fairness more difficult.

\textbf{Conceptual description:}
\begin{enumerate}
    \item For each STA $k \in \mathcal{C}$, find the set of STAs who are semi-orthogonal to $k$ such that $\mathcal{S}_k = \lbrace j\ : \ \vert \textbf{h}_k \textbf{h}_j^H \vert < \alpha \rbrace \ \forall j \neq k,\ k = 1,2,3\ldots\vert\mathcal{C}\vert$
    \item Build a tree for each $k$ that represents potential SUS groups that contain $k$
    \item Form SUS group sets based on the traversal of each tree
    \item Remove all SUS group sets that are larger than the number of antennas available at the AP
    \item Form $\mathfrak{S}$ by permuting the sets produced by tree traversals while making sure that STAs are not duplicated in groups.
\end{enumerate}

Each group of semi-orthogonal STAs belonging to $k$, $\mathcal{S}_k$ is formed into a tree. The trees are structured using the following algorithm.
\begin{enumerate}
    \item Place $k$ at the root of the tree
    \item Write the first STA $l \in \mathcal{S}_k$ as the child of $k$.
    \item Repeat the previous step adding children to the current branch of the tree so long as the orthogonal sets intersect with all of the parent nodes of the tree. Add a child node to the branch for every STA $j : j \in \mathcal{S}_k \cap j \in \mathcal{S}_l$. Once $j$ has been added to the tree remove $j$ from $\mathcal{S}_k$.
    For example, given three STAs $A,B,C$, then $A$'s tree will be three deep if $B\in\mathcal{S}_A \cap C\in \mathcal{S}_A \cap C \in \mathcal{S}_B$. In other words, A,B,C are mutually semi-orthogonal to each other.
\end{enumerate}

SUS groups containing $k$ are generated by permuting the nodes that exist on any branch between the root and leaf node. The permutations must include the root of the tree. Permuting the traversals of a tree generated from $\mathcal{S}_k$ will give all the possible SUS groups that contain $k$.

Finally, each of the SUS groups generated for a given STA are permuted with each other to generate $\mathfrak{S}$. Assuming that each STA must be included in $\mathfrak{S}$ once and only once, the SUS groups generated from a tree are muted once the STA at the root of that tree is included in a previous SUS group.
	
    
    \newpage	
 	\begingroup
 		\renewcommand{\section}[2]{}%
 		\bibliographystyle{IEEEtran}
 		\bibliography{user_sel_lit}
 	\endgroup
\end{document}
