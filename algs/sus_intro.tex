Solutions to the semi-orthogonal user selection (SUS) problem can be roughly classified into three categories. The first category is the identification of a single SUS group. In this context, assume an AP is capable of providing service to a group of STAs represented by a set $\mathcal{C}$. A SUS group, $\mathcal{S} \subset \mathcal{C}$ may only contain part of $\mathcal{C}$. Therefore there are many such solutions that exist in $\mathcal{C}$. A second solution the SUS problem is by finding multiple SUS groups such that the collection of these groups partition the set $\mathcal{C}$. That is, the partition, $\mathscr{S}$ is composed of SUS groups $\mathcal{S}_l\ :\ l = 1,2,\ldots L$ such that it fully partitions $\mathcal{C}$. Similar to the first case, this solution is not unique. There may be many such partitions $\mathscr{S}$ of $\mathcal{C}$. Finally, an exhaustive solution to the problem may be conceived by describing all the potential partitions of $\mathcal{C}$. That is, the collection of all possible partitions on $\mathcal{C}$, $\mathfrak{S}$, is a set of partitions $\mathscr{S}_n\ : \ n=1,2,\ldots N$ on $\mathcal{C}$, which, in turn, is a set of SUS groups $\mathcal{S}_l \ :\ l = 1,2,\ldots L$. The size of the sets, $L,N$, depend on $\vert \mathcal{C}\vert$ and the strictness of the orthogonality requirement between SUS groups.

