The algorithm developed in \cite{1603708} only identifies one group: it does not identify all the potential SUS groups. Therefore, an extension of this algorithm is required for our study. At minimum, we must have a set of viable SUS groups that belong to one AP. A more exhaustive search would return all the viable sets of SUS groups. 

The utility of such an exhaustive search is limited, especially when considering the computational cost of executing such a search.
Assume the following definitions:
\begin{itemize}
    \item Let $\mathcal{C}$ denote the set of STAs available to form SUS groups
    \item Let $K = \vert \mathcal{C} \vert $
    \item Let $k \in \mathcal{C}$ be an arbitrary STA
    \item Let $M = $ the number of antennas at an arbitrary AP, $a$
    \item Let $\textbf{h}_k$ be the $M \times 1$ channel vector for $k$
    \item Let $i$ be the variable of iteration
    \item Let $\mathcal{T}_i$ be the set of STAs who are semi-orthogonal candidates to be added to an SUS group at the $i^{th}$ iteration
    \item Let $\mathscr{S}$ be a partition on $\mathcal{C}$. It is a set containing the SUS groups
    \item Let $\mathcal{S}_l \in \mathscr{S}$ be the $l^{th}$ sub-optimal SUS group. The elements of $\mathcal{S}$ are STAs.
    \item Let $\alpha$ be the scalar representing the orthogonality requirement of the SUS group
\end{itemize}


\textbf{Algorithm from \cite{1603708}:}
\begin{enumerate}
    \item Initialization:
    \begin{subequations}
        \begin{align}
            \mathcal{T}_1 &= \lbrace 1 \ldots K \rbrace = \mathcal{C}\\
            i &= 1\\
            \mathcal{S}_1 &= \lbrace \emptyset \rbrace
        \end{align}
    \end{subequations}
    Start by populating the candidate set with the entire set of STAs available to the AP. Initialize the SUS group to be an empty set.
    \item Calculate orthogonal component of $\textbf{h}_k$:\\
    For $\forall \ k \in \mathcal{T}_i$:
    \begin{subequations}
        \begin{align}
            \textbf{g}_k &= \textbf{h}_k - \sum_{j=1}^{i-1}\frac{\textbf{h}_k\textbf{g}_j^*}{\Vert \textbf{g}_j \Vert^2}\textbf{g}_j \\
            &= \textbf{h}_k\bigg(\textbf{I}-\sum_{j=1}^{i-1}\frac{\textbf{g}_k^*\textbf{g}_j}{\Vert \textbf{g}_j \Vert^2}\bigg)
        \end{align}
    \end{subequations}
    This expression represents the projection of the channel vector $\textbf{h}_k$ onto the orthogonal complement of the span of space formed by $\text{span}(\textbf{g}_1\ldots \textbf{g}_{i-1})$. Initially, $\textbf{g}_k = \textbf{h}_k \ \forall \ k \in \mathcal{T}_1$. Then as the algorithm progresses, $\textbf{g}_k$ represents an inverse degree of orthogonality. This can be seen in Eq. (2a): as the second term $\longrightarrow 0$, representing orthogonality between $\textbf{h}_k$ and $\textbf{g}_j$, $\textbf{g}_k$ gets large.
    
    \item Select the best STA representing the highest degree of orthogonality:
    \begin{subequations}
        \begin{align}
            \pi(i) &= \arg\max_{k\in \mathcal{T}_i}\Vert \textbf{g}_k \Vert\\
            \mathcal{S}_1 &= \mathcal{S}_1 \cup \pi(i)\\
            \textbf{h}_i &= \textbf{h}_{\pi(i)}\\
            \textbf{g}_i &= \textbf{g}_{\pi(i)}
        \end{align}
    \end{subequations}
    In this step, the STA with the hightest degree of orthogonality to the remaining candidates is selected as the next STA to be added to the SUS group, $\mathcal{S}_o$. Channel and orthogonality vectors are updated to reflect this selection.
    
    \item Update the set of candidate STAs based on semi-orthogonality constraint:\\
    If $\vert \mathcal{S}_1 \vert < M$, then:
    \begin{subequations}
        \begin{align}
            \mathcal{T}_{i+1} &= \bigg\lbrace k\in \mathcal{T}_i,\ k\neq \pi(i)\ \big \vert \ \frac{\vert \textbf{h}_k\textbf{g}_i^*\vert}{\Vert \textbf{h}_k \Vert \Vert \textbf{g}_i\Vert} < \alpha  \bigg\rbrace \\
            i &= i+1
        \end{align}
    \end{subequations}
    Else: Terminate algorithm.
    The set of candidate STAs is updated for the next iteration of the algorithm by  removing any STAs that do not meet a minimum orthogonality requirement, determined by $\alpha$.
\end{enumerate}


\textbf{Extension of Algorithm}\\
The algorithm developed here extends the SUS algorithm developed in \cite{1603708}. The algorithm here takes the set of viable STAs, $\mathcal{C}$, the channel information for each STA $k \in \mathcal{C}$, $\textbf{h}_k$, and a orthogonality threshold, $\alpha$. The algorithm takes this information and partitions $\mathcal{C}$ into semi-orthogonal subsets, $\mathscr{S}$.

The solution provided by this algorithm is only one of many potential partitions on $\mathcal{C}$. That is, the solution provided here is not exhaustive, nor is it unique.

The algorithm essentially wraps the algorithm developed in \cite{1603708} in additional loop that checks to see that all the STS $\in \mathcal{C}$ have been assigned to a SUS group. The algorithm does not guarantee that all groups are the same size; however, it does guarantee that the SUS group size is less than the number of antennas at the AP.

\begin{algorithm}[H]
 \KwData{\\$\mathcal{C}$ (set of viable STAs);\\$\textbf{h}_k,\ \forall \ k\in\mathcal{C}$ (channel information for each STA);\\
 $\alpha$ (orthogonality requirement)}
 \KwResult{$\mathscr{S}$ (SUS partition on $\mathcal{C}$)}
 initialization;\\
        $\mathcal{L} = \lbrace 1 \ldots K \rbrace = \mathcal{C}$;\\
        $\mathcal{T}_1 = \lbrace 1 \ldots K \rbrace = \mathcal{C}$;\\
        $\mathcal{S}_1 = \lbrace \emptyset \rbrace$;\\
        $i = 1$;\\
        $l = 1$;\\

 \While{$\mathscr{L} \neq \lbrace \emptyset \rbrace$}{
     \While{$\vert \mathcal{S}_l \vert < M$ and $\mathcal{T}_i\neq\lbrace \emptyset \rbrace$}{
        \For{$\forall\ k \in \mathcal{T}_i$}{
            $\textbf{g}_k = \textbf{h}_k - \sum_{j=1}^{i-1}\frac{\textbf{h}_k\textbf{g}_j^*}{\Vert \textbf{g}_j \Vert^2}\textbf{g}_j$; \\
            %&= \textbf{h}_k\bigg(\textbf{I}-\sum_{j=1}^{i-1}\frac{\textbf{g}_k^*\textbf{g}_j}{\Vert \textbf{g}_j \Vert^2}\bigg)
        }
        $\pi(i) = \arg\max_{k\in \mathcal{T}_i}\Vert \textbf{g}_k \Vert$;\\
        $\mathcal{S}_1 = \mathcal{S}_l \cup \pi(i)$;\\
        $\textbf{h}_i = \textbf{h}_{\pi(i)}$;\\
        $\textbf{g}_i = \textbf{g}_{\pi(i)}$;\\
        //Add remaining STA to candidate set for SUS group if it meets SUS threshold\\
        //Otherwise add to the leftover set $\mathcal{L}$\\
        \For{$\forall \ k \in \mathcal{T}_i$,$\ k\neq \pi(i)$}{
            \eIf{$\frac{\vert \textbf{h}_k\textbf{g}_i^*\vert}{\Vert \textbf{h}_k \Vert \Vert \textbf{g}_i\Vert} < \alpha$}{
                $k \in \mathcal{T}_{i+1}$;\\
                $k \notin \mathcal{L}$;\\
            }{
                $k \in \mathcal{L}$;\\
                $k \notin \mathcal{T}_{i+1}$;\\
            }  
        }
        $i = i + 1$;\\
    }
    $l = l+1$;
 }
 \caption{Semi-orthogonal user selection algorithm}
\end{algorithm}