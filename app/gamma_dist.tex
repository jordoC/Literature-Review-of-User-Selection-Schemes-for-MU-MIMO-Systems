
 Given a standard normal random variable, $X_k\sim\mathcal{N}(0,1)$, corresponding realization, $x_k$, probability density function (PDF) $f_Xk(x_k)$, and cumulative distribution function (CDF) $F_Xk(x_k)$:

\begin{equation}\label{eq:normal}
    \begin{aligned}
        f_N(x;\mu,\sigma) &\triangleq \frac{1}{\sqrt{2\pi\sigma^2}}e^{-\frac{(x-\mu)^2}{2\sigma^2}};\\
        f_{Xk}(x_k) &= f_N(x_k;0,1);\\
        F_{Xk}(x_k) &=\frac{1}{2}(1+\text{erf}(\frac{x_k}{\sqrt{2}}))
    \end{aligned}
\end{equation}

Let $\textbf{X}$ denote an iid vector of length $K$. Each element of $\textbf{X}$ follows the standard normal distribution given in Eq. (\ref{eq:normal}), that is, $X_k\in\textbf{X}\ \forall k = 1,2\ldots K$.

Let $Y$ be the random variable representing the L2 norm of the vector $\textbf{X}$:
\begin{equation}\label{eq:ch_sq_sum}
    \begin{aligned}
        Y &= \Vert \textbf{X} \Vert^2\\
          &= \sum_{k = 1}^K X_kX_k^*
    \end{aligned}
\end{equation}

It is well-known that a K-length sum of square standard normal random variables results in a Chi-squared distribution.The PDF and CDF of the Chi-squared distribution associated with $Y$ in terms of realizations $y = \sum_{k=0}^K x_k x_k^*$ are given by $f_Y(K,y)$, and $F_Y(K,y)$, respectively:
\begin{equation}\label{eq:ch_sq}
    \begin{aligned}
        f_{\chi^2}(y;k) &\triangleq  \frac{1}{2^{k/2}\Gamma(k/2)}y^{\frac{k-2}{2}}e^{\frac{-y}{2}};\\
        f_Y(y;K) &= f_{\chi^2}(y;K);\\
        F_Y(y;K) &= \frac{\gamma(K/2,y/2)}{\Gamma(K/2)}\\
        &= \Gamma_n(K/2,y/2)
    \end{aligned}
\end{equation}

Now we wish to investigate the case where the normal distributions are no longer standard. Rather, they are zero-mean normal distributions with variance $\sigma^2$. Let $Z$ denote the scaled sum:
\begin{equation}\label{eq:ch_sq_sum_scaled}
    \begin{aligned}
        Z &= \Vert \sigma \textbf{X} \Vert^2\\
          &= \sigma^2 \sum_{k = 1}^K X_kX_k^*
    \end{aligned}
\end{equation}

In this case we arrive at a Gamma distribution rather than the special case of the Chi-squared distribution. The PDF and CDF become:
\begin{equation}\label{eq:gamma_scaled}
    \begin{aligned}
        f_\Gamma(\gamma; n,\theta) &\triangleq \frac{\gamma^{n-1}}{\Gamma(n)\theta^n}e^{\frac{-\gamma}{\theta}} \ \forall \gamma > 0;\\
        %&= \frac{1}{\sigma^{K/2}2^{K/2}\Gamma(K/2)}z^{\frac{K-2}{2}}e^{\frac{-z}{2\sigma^2}};\\
        f_Z(z; K, \sigma) &= f_\Gamma(z;K/2,2\sigma^2);\\
        F_Z(z;K,\sigma) &= \Gamma_n(K/2,z/(2\sigma))
    \end{aligned}
\end{equation}
where $k$ and $\theta$ are the conventional parameters used to parametrize the PDF of the gamma function, $f_Z(z;k,\theta)$.\\

We  now examine the final case where we have $k$ zero-mean random variables, each of which has a (potentially) unique variance with respect to other normal random variables. Let $\Tilde{X_k}\sim\mathcal{N}(0,\sigma_k^2)\ \in \Tilde{\textbf{X}}$ Forming the sum of squares results in the following expression:
\begin{equation}\label{eq:gamma_sum_indep_scaled}
    \begin{aligned}
        \Tilde{Z} &= \Vert \Tilde{\textbf{X}} \Vert^2 \\
        &= \sum_{k = 1}\Tilde{X_k}\Tilde{X_k^*}\\
        &= \sum_{k = 1}^K \sigma_k^2X_kX_k^*
    \end{aligned}
\end{equation}

In order to arrive at an expression for the PDF and CDF for the random variable $\Tilde{Z}$, we first consider the argument of the sum. We know that $\Tilde{X_k}\Tilde{X_k} = \vert \Tilde{X_k}\vert^2\sim\mathcal{G}(1/2, 2\sigma_k^2)$, where $\mathcal{G}(n,\theta)$ denotes a Gamma-distributed random variable with PDF $f_\Gamma(n,\theta)$. Therefore, the sum in Eq. (\ref{eq:gamma_sum_indep_scaled}) becomes a sum of $K$ independent Gamma-distributed random variables. We define the parameter vectors: $\theta_k \in \boldsymbol{\theta},\ n_k \in \textbf{n} \ \forall k = 1,2,\ldots K$ that contain the parameters for each of the $K$ random variables in the sum forming $\Tilde{Z}$. Investigation into developing expressions for such a sum has been conducted in \cite{Moschopoulos1985} and \cite{Mathai1982}; we adopt the recursive formulation developed in the prior. The expression for the PDF of $\Tilde{Z}$ is given as follows:\\
\begin{equation}\label{eq:gamma_sum_indep_scaled_pdf}
    \begin{aligned}
        f_{\Tilde{Z}}(\Tilde{z};\boldsymbol{\theta},\textbf{n}) &= C\sum_{l = 0}^\infty \delta_l \Tilde{z}^{\zeta+l-1}\frac{e^{-\Tilde{z}/\theta_1}}{\Gamma(\zeta+l)\theta_1^{\zeta+l}} \ \forall \Tilde{z} > 0; \\
        where:\\
        \theta_1 &= \min_k(\theta_k) \ \forall k=1,2,\ldots K;\\
        \zeta &= \sum_{k = 1}^K n_k;\\
        \alpha_p &= \sum_{k = 1}^K \frac{n_k(1-\theta_1/\theta_k)^p}{p}\\
        C &= \prod_{k = 1}^K \frac{\theta_1}{\theta_k}^{n_k};\\
        \delta_{l+1} &= \frac{1}{l+1}\sum_{p = 1}^{l+1} p\alpha_p\delta_{l+1-p};\\
        \delta_0 &= 1
    \end{aligned}
\end{equation}