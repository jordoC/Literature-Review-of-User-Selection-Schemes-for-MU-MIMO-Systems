The focus of this study is motivated by developing an alternative to dirty paper coding (DPC) that is less computationally expensive. It is postulated that DPC is an optimal scheme in terms of throughput maximization in so far as its reduction of interference between users in a MU-MIMO network. The drawback of this coding scheme is it's high computational complexity. Authors present zero-forcing beamforming (ZFBF) scheme in coordination with semi-orthogonal user selection scheme as a computationally simplistic alternative to DPC.

Authors argue that as the number of users grows, the performance of SUS-ZFBF approaches DPC. The reason for this asymptotic relationship are two-fold. Firstly, assuming a Rayleigh-fading channel, the transmitter has more choices to avoid poor channel conditions. The user with the best SNR exhibits an SNR a factor of $\text{log}(K)$ higher than the average user SNR (where $K$ is the number of users present in the environment). Secondly, the diversity provided by user separation offers increased diversity. Separation of users provides more isolation between users, resulting in lower interference, and more spatially independent choices for the user to transmit to, thus enhancing overall capacity.

Paper develops good channel model. Rates for TDMA, DPC, and ZFBF schemes are well-developed.

It is identified that optimal SUS requires an exhaustive search on the set of STAs present. This search becomes in feasible as this set gets large. Therefore, the authors offer an algorithm with reduced complexity to find a sub-optimal SUS scheme. The scheme presented depends on an orthogonality variable, $\alpha$, that permits a degree of flexibility when developing SUS groups (similar to the $\epsilon$ variable in Swannack \cite{1549555}). The algorithm presented only identifies one SUS group. Extension of this algorithm is required for our work. 

Note that \cite{1603708} goes on to develop a ZFBF scheme based on the SUS parameters. The ZFBF scheme presented decomposes the channel matrex of the SUS group, $\textbf{H}(\mathcal{S})$, into diagonal, lower triangle, and unitary matrices which depend on $\alpha$, $\textbf{g}_k$. Once the decomposition is performed, a realizable ZFBF scheme is presented. Performance is lower-bounded based on orthogonality arguments.

Complexity analysis is provided, which is likely useful in justifying a SUS scheme opposed to other schemes (such as DPC, TDMA etc.).

Results show that there is a convex relationship between $\alpha$ and sum rate. This makes sense since there is a sweet spot when trading off quantity vs. quality associated with $\alpha$.

The work goes on to consider fairness in its scheduling schemes. This will likely be of interest at some point.
